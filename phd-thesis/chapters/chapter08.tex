%==============================================================================
\chapter{In silico quantitatively mapping force-pCa curves to whole heart contraction and relaxation}\label{cha:chapter8}
%==============================================================================
%
%
%
\begin{remark}{Outline}
    In this chapter, we investigate the force-calcium relationship using the cell contraction sub-model of our full personalised SHAM rat hart contraction model (Section~\ref{sec:cellcontr}). Using a combination of simulator and emulators (Section~\ref{subsec:quantlink}), we show that changes in key features of the F-pCa curve cannot be uniquely described by changes in key sarcomere properties (Sections~\ref{sec:changespCa50result1}--\ref{sec:changespCa50result2}). At the same time, we show that changes in the LV function (e.g. ejection fraction) cannot be uniquely described by changes in the sarcomere properties (Section~\ref{sec:changesLVfunctionresult}). By coupling these two pieces of information, we demonstrate that the mapping from the F-pCa curve to the LV function and the corresponding inverse mapping are non-unique (Section~\ref{sec:ch8nonuniquemappings}). We then provide a discussion and address specific limitations (Section~\ref{sec:ch8discussion}), and we conclude with a brief summary (Section~\ref{sec:ch8summary}).
\end{remark}


%
%
%
\section{Motivation}\label{sec:ch8motivation}
Heart failure (HF) affects nearly a million people in the UK alone~\cite{Bhf:2021}. The gold standard pharmacotherapies for heart failure with reduced ejection fraction (EF) phenotype (HFrEF) are beta-blockers, angiotensin receptor/neprilysin inhibitors, sodium-glucose co-transporter inhibitors, beta-blockers and mineralocorticoid receptor antagonists. These were shown to improve the primary endpoint composite of cardiovascular death and HF hospitalisation in breakthrough clinical trials~\cite{Debska-Kozlowska:2021}. Nevertheless, improved treatment strategies for HFrEF and for HF with preserved ejection fraction (HFpEF) are constantly being sought. The Food and Drug Administration has recently stated~\cite{FDA:2019} that an improvement in symptoms and/or physical function, even without a documented favourable effect on the above mentioned primary endpoints, can be the basis for a drug to be approved as HF treatment~\cite{Fiuzat:2020}. For this reason, physiopathologic or `explanatory' endpoints~\cite{Zanolla:2003} have been regaining attention in the initial drug assessment phase and as secondary endpoints in later phases of clinical trials. For instance, non-invasive measurements of the left ventricular (LV) function such as cardiac output fall within these categories as targets for improvement (e.g. increased stroke volume (SV) / EF)~\cite{Zanolla:2003}.

\vspace{0.2cm}
Novel strategies of HF treatment are being developed that tend to differ from the above-mentioned pharmacotherapies that focus on the neurohormonal modulation paradigm. Among novel strategies, the direct sarcomere modulators are an important compound class in this direction~\cite{Tsukamoto:2020}. Sarcomere modulators' mechanism of action relies on the fact that sarcomeric proteins dynamics are the basis of myocardial contraction and relaxation~\cite{Solaro:1998}. The two new compounds omecamtiv mecarbil~\cite{Teerlink:2016} and mavacamten~\cite{Olivotto:2020} provide new sarcomere modulators that can increase and decrease sarcomere activation by cardiac myosin stimulation and inhibition, respectively. Both these compounds alter myofilament calcium ($\Ca$) sensitivity, and knowing when and why this will result in improved cardiac function is an important question.

\vspace{0.2cm}
One potential assessment for sarcomere modulators is their impact on the \textit{force-pCa} (F-pCa) \textit{curve}~\cite{Walker:2010}, which is a technique that is widely used in many HF diseases including rare/genetic diseases such as dilated and hypertrophic cardiomyopathies~\cite{Groen:2020,Bai:2013,Michael:2016,Kirschner:2005,Harris:2002}, as well as for HF with reduced and preserved ejection fraction~\cite{Nagy:2015,Kampourakis:2018,Kieu:2019,Awinda:2021,Mamidi:2018,Sparrow:2020}. The impact on the F-pCa curve is often examined in terms of observed shifts in the curve half-maximal activation ($\pCaf$), which is used as an index of $\Ca$ sensitivity. Leftward shifts in the $\pCaf$ are expected to improve contractility, whereas rightward shifts to decrease contractility. Sometimes there is an assumption that changes in $\Ca$ sensitivity are a surrogate for changes in whole heart function and that these changes are monotonic in that, for instance, an increase in myofilament $\Ca$ sensitivity (i.e. a leftward shift) would improve whole heart cardiac output. In this paper, we discuss the limitations of such an assumption and demonstrate why the dynamic whole heart behaviour cannot be predicted or assessed with a shift in the F-pCa curve alone.

\vspace{0.2cm}
We propose to use a mathematical model of active tension generation at the sarcomere level in LV rat myocytes to elucidate the relationship between sarcomere properties and the steady-state F-pCa curve. We then integrate the cell tension model within a $3$D biventricular rat heart contraction model to provide a quantitative link between sarcomere properties and whole heart contractile function. In particular, with this model, we are able to quantitatively link changes in the F-pCa curve to/from changes in the LV function for the first time to the best of our knowledge. We use this \textit{in-silico} framework to show that observations made at the sarcomere level (e.g. a shift in the F-pCa curve) do not uniquely map to desired effects at the whole-organ level (e.g. an increase in EF or improved relaxation). We also show that the opposite holds: given a change in LV function, there exist many ways this could have been achieved through sarcomeric modulation. The implications of these phenomena on using F-pCa curves to interpret whole heart dynamics is discussed in Section~\ref{sec:ch8discussion}.


%
%
%
\section{Methods}\label{sec:ch8methods}


%
%
%
\subsection{Cellular contraction model}\label{sec:cellcontr}
We employed the Land et al.~\cite{Land:2012} myocyte contraction model (Section~\ref{sec:ch2contractionmodel}) to simulate active tension generation at the sarcomere level and isometric steady-state F-pCa relationship in the rat heart. Parameters were all set to the personalised SHAM rat model values (Table~\ref{tab:bestfitparametersvalues}) when applicable, or to the Land et al.~\cite{Land:2012} model baseline values when otherwise.

\vspace{0.2cm}
We recall that an experimentally derived F-pCa curve can be described by a Hill-type relationship between the force ($F$) and the negative logarithm of the $\Ca$ concentration ($\pCa$)~\cite{Walker:2010}:
%
\begin{equation}
    \frac{F}{F_0} = \frac{1}{1+10^{h(\pCaf-\pCa)}}
\end{equation}

\noindent
where $F_{0}$ is the maximal reference force and $\pCaf$ and $h$ are respectively the negative logarithm of the half maximal effective $\Ca$ concentration and the Hill coefficient of this relationship. It immediately follows that F-pCa curves are uniquely determined by the $\pCaf$ and $h$ values. We have also seen that the steady-state force solution of the Land et al.~\cite{Land:2012} model can be derived analytically, with both the $\pCaf$ and $h$ features expressed in terms of model parameters. In particular, we recall that the $\pCaf$ was given as:
%
\begin{align}
    & \pCaf(\mathbf{p}) =  -\log\left[\Caif\left(\frac{\koff}{\kon}\frac{\trpnf}{1-\trpnf}\right)^{1/\ntrpn}\right]\,,\quad\text{with} \label{eq:pCa50}\\
    & \mathbf{p} := (\Caif, \kon, \koff, \trpnf, \ntrpn) \label{eq:pCa50params}
\end{align}

\vspace{0.2cm}\noindent
In the next section, we shall make use of equations~\eqref{eq:pCa50}--\eqref{eq:pCa50params} to investigate on F-pCa curve shifts. 


%
%
%
\subsection{Non-unique mapping of changes in $\pCaf$ to changes in sarcomere properties}\label{sec:changespCa50}
A given change ($\Delta$) in the $\pCaf$ feature value of the F-pCa curve can be written as:
%
\begin{equation}\label{eq:delta}
    \Delta = \pCaf^{\text{new}} - \pCaf
\end{equation}

\vspace{0.2cm}\noindent
where $\pCaf^{\text{new}}$ is the feature value characterising the newly observed F-pCa curve. We tested whether a shift of $\Delta$ units in the $\pCaf$ could be the result of unique changes in sarcomere properties. Testing for non-unique changes was carried out analytically and numerically (as visualised in Figure~\ref{fig:2perclwrwshift}) in the case where the change was driven by perturbations in one model parameter (Section~\ref{sec:changespCa50result1}), and numerically (as visualised in Figure~\ref{fig:pca50isolines}) in the case where this was driven by perturbations in two model parameters (Section~\ref{sec:changespCa50result2}).

\vspace{0.2cm}
In the one-parameter case, it sufficed to prove that for each parameter $p_i,\,i=1,\,\dots,\,5$ of equation~\eqref{eq:pCa50params} there exists a scaling coefficient $\alpha_i\in\mathbb{R}$ such that
%
\begin{equation}
    \mathbf{p}^{\text{new}} := (p_1,\,\dots,\,\p_{i-1},\,\alpha_i\times\p_i,\,p_{i+1},\,\dots,\,p_5)
\end{equation}

\noindent
is such that
%
\begin{equation}
    \pCaf^{\text{new}} = \pCaf(\mathbf{p}^{\text{new}})
\end{equation}

\vspace{0.2cm}\noindent
We computed this scaling coefficient for each parameter $\p_i$ by solving for $\alpha_i$ equation~\eqref{eq:delta} for a given $\Delta$.

\vspace{0.2cm}
In the two-parameter case, we proceeded as follows. We allowed $2$ parameters to take equally-spaced values within a $\pm\SI{50}{\percent}$ perturbation around their baseline values (Table~\ref{tab:bestfitparametersvalues}). We then generated a $2$D uniform grid from all the combinations of $2$ parameter values, and used the Land et al.~\cite{Land:2012} cell contraction model to calculate the $\pCaf$ value at each parameter point of the grid. By plotting the resulting $\pCaf$ values across the whole grid as a heat map, we could discern regions in the $2$D parameter space which share the same $\pCaf$ value (isolines). The same process was repeated for pairs of parameters coming from vector $\mathbf{p}$ (equation~\eqref{eq:pCa50params}), which was shown to regulate the $\pCaf$ feature of the F-pCa curve (equation~\eqref{eq:pCa50}).


%
%
%
\subsection{The relationship between sarcomere properties and LV function}\label{subsec:quantlink}
To quantitatively link sarcomere properties to the LV function, we employed the personalised model of $3$D biventricular healthy (SHAM) rat heart contraction derived in Chapter~\ref{cha:chapter4} and extended in Chapter~\ref{cha:chapter7}. This model integrates the Land et al.~\cite{Land:2012} cell contraction model in the context of whole-organ simulations.

\vspace{0.2cm}
We have seen (Section~\ref{sec:ch7rat_heart_contraction_model}) that this model can be represented as a multi-scale function (simulator) that maps a $16$D vector of input parameters $\mathbf{x}$ to a scalar quantity of interest $y_j$ (e.g. $y_1=\textrm{EDV}$, $y_2=\textrm{ESV}$, etc.):
%
\begin{align}\label{eq:fsimul}
    f_{\textrm{simul}}\colon\mathbb{R}^{16} &\to\mathbb{R}\times\mathbb{R}\times\dots \\
    \mathbf{x} &\mapsto (y_1,\,y_2,\dots) \nonumber
\end{align}

\vspace{0.2cm}\noindent
At the same time, we have replaced (Section~\ref{sec:ch7training_dataset_emulators_global_sensitivity_analysis}) the simulator with a fast-evaluating surrogate model (emulator), based on Gaussian process emulation.

\vspace{0.2cm}
We used the simulator to investigate the dependence of LV volume features, namely EDV, ESV, SV, and EF, on model parameters. Specifically, we were interested in the dependence of the features on the $\Caif$, $\koff$, $\ntrpn$ and $\trpnf$ parameters as these directly modulated the $\pCaf$ feature of the F-pCa curve (equation~\eqref{eq:pCa50}). For each parameter, we performed the following operations. First, the simulator was run at $128$ input points $\mathbf{x}_i,\,i=1,\dots,128$, which had all their $16$ components fixed to baseline values (Table~\ref{tab:bestfitparametersvalues}) but the component corresponding to the examined parameter, which instead was set to $128$ equally-spaced values between $-\SI{50}{\percent}$ and $\SI{50}{\percent}$ of perturbation values from baseline. The LV volume features were then extracted from the simulated $128$ LV volume transients so that we could plot the variation of each feature as a function of the parameter variations around its baseline value.

\vspace{0.2cm}
We then used the emulator to perform a GPE-based GSA. We calculated the Sobol' first-order, second-order and total effects (Section~\ref{sec:ch3globalsensitivityanalysis}) as a measure of model outputs' sensitivity to model inputs. We evaluated the impact of $\Caif$, $\koff$, $\ntrpn$, $\trpnf$ parameters into affecting the total variance of EDV, ESV, SV, EF features. Parameters whose Sobol' indices' distributions' expectation was below the threshold $0.01$ were determined to have negligible effects.


%
%
%
\subsection{Non-unique mapping of changes in LV function to changes in sarcomere properties}\label{sec:changesLVfunction}
Left ventricular EF can be continuously monitored noninvasively as an index of LV systolic function in patients enrolled for clinical trials investigating on HF treating drugs. However, it is important to discern whether detected changes in the EF value are the direct result of a specific cellular modulation caused by the administered drug or not. From a modelling perspective and in the context of sarcomere modulators, we can test whether changes observed in the LV function at whole-organ level, as described by the EF feature, can be uniquely explained in terms of sarcomere properties' changes.

\vspace{0.2cm}
We therefore proceeded as in Section~\ref{sec:changespCa50} by generating a $2$D uniform grid of parameter values for each considered pair of parameters modulating the F-pCa curve (equation~\eqref{eq:pCa50}). We then used the emulator to predict the EF value at each parameter point of the grid. By plotting the resulting EF values across the whole grid as a heat map, we could again discern regions in the $2$D parameter space which share the same EF value (isolines).


%
%
%
\subsection{Non-unique mapping of F-pCa curve to LV function and of LV function to F-pCa curve}
In order to test if the mapping from the F-pCa curve to LV function is not unique, we considered all the parameter points of the $2$D grids generated in Section~\ref{sec:changespCa50} that produced the same shift in the reference $\pCaf$ feature value. As mentioned previously, these points all belong to isolines of the local $2$D parameter space. We then used the emulator to map the points to the corresponding organ-level EF value.

\vspace{0.2cm}
We also tested if the inverse mapping from the LV function to the F-pCa curve is not unique. For this purpose, we considered all the parameter points of the $2$D grids generated in Section~\ref{sec:changesLVfunction} that produced the same shift in the reference EF feature value. We then used the cellular contraction sub-model of the full simulator to map the points to the corresponding cell-level $\pCaf$ value.


%
%
%
\section{Results}


%
%
%
\subsection{Non-monotonic relationship between sarcomere parameters and LV function}
The variations of EDV, ESV, SV and EF features as a function of the $4$ considered F-pCa curve-modulating parameters (equation~\eqref{eq:pCa50}) are non-linear and non-monotonic, as shown in Figure~\ref{fig:EFvsparamsnonmonotonic}. An example of full simulator outputs comprising LV volume and pressure transients and PV loops is provided in Figure~\ref{fig:EFvskoff} for the EF-vs-$\koff$ case.

\begin{figure}[h!]
    \myfloatalign
    \includegraphics[width=\textwidth]{figures/chapter08/Fig1.pdf}
    \caption{The full $3$D biventricular rat heart contraction model is run at a fixed parameter set with only one parameter taking equally-spaced values in the $\pm\SI{50}{\percent}$ range of perturbation from its baseline value (vertical red dashed lines). The converging mechanics simulations' output PV loops are analysed to extract the corresponding EDV, ESV, SV and EF features' values (open blue dots), given as percentages from their baseline values (horizontal red dashed lines). The process is repeated separately for each parameter regulating the $\pCaf$ feature of the F-pCa curve. A linear regression (LR) model with second-order degree polynomials is fitted to the data (blue lines) to facilitate visualisation of non-linear and non-monotonic relationships between the features and each of the parameters considered.}
    \label{fig:EFvsparamsnonmonotonic}
\end{figure}

\begin{figure}[h!]
    \myfloatalign
    \includegraphics[width=\textwidth]{figures/chapter08/Fig2.pdf}
    \caption{The full $3$D biventricular rat heart contraction model is run at a fixed parameter set with only one parameter varying around its baseline value. Resulting LV volume and pressure transients and corresponding PV loops are plotted for some of the parameter values (full lines in blue variants), and compared to the reference parameter set mechanics solution (dashed black line). Example showing $\koff$ parameter variation in the range obtained as a $\pm\SI{40}{\percent}$ perturbation of its baseline value.}
    \label{fig:EFvskoff}
\end{figure}

\vspace{0.2cm}
The obtained GSA Sobol' sensitivity indices are displayed as donut charts in Figure~\ref{fig:gsarestr}. We can see that the reference thin filament $\Ca$ sensitivity ($\Caif$) is the most important parameter in explaining the total variance of EDV, ESV, SV and EF features. The second most important parameter is the degree of cooperativity of $\Ca$ binding to TnC ($\ntrpn$), followed by the fraction of bound $\Ca$-TnC complexes for half maximal cross-bridges activation ($\trpnf$) and the unbinding rate of $\Ca$ from TnC ($\koff$).

\begin{figure}[h!]
    \myfloatalign
    \includegraphics[width=\textwidth]{figures/chapter08/Fig3.pdf}
    \caption{The impact of $\pCaf$-modulating, sarcomere parameters on EDV, ESV, SV and EF organ-scale LV feature. The contribution of each parameter is represented by the sum of its first- and second-order effects. For each LV feature, higher-order interactions (up to the fourth order) are represented by the sum of all total effects minus the sum of all first- and second-order effects.}
    \label{fig:gsarestr}
\end{figure}


%
%
%
\subsection{$\pCaf$ changes are non-uniquely explained by sarcomere alterations: the one-parameter case}\label{sec:changespCa50result1}
We show that an observed change in $\pCaf$ can be caused by multiple different changes in sarcomere proprieties represented by model parameters. For each parameter $p_i,\,i=1,\,\dots,\,5$ of equation~\eqref{eq:pCa50params}, we report in Table~\ref{tab:alphavalues} the corresponding scaling coefficient $\alpha_i,\,i=1,\,\dots,\,5$ that would yield a shift of exactly $\Delta$ units in the $\pCaf$ value. This proves that an observed change in the F-pCa cannot be uniquely explained by a change in a specific sarcomere property. An example of $\pm\SI{2}{\percent}$ shift from a reference $\pCaf$ value is displayed in Figure~\ref{fig:2perclwrwshift}.

\begin{table}[h!]
    \myfloatalign
    \begin{tabularx}{\textwidth}{XX}
        \toprule
        \tableheadline{Parameter} & \tableheadline{Scaling coefficient $\alpha$} \\
        \midrule
        & \\
        $\Caif$ & $10^{\Delta}$ \\ & \\
        $\kon$ & $10^{-\Delta\cdot\ntrpn}$ \\ & \\
        $\koff$ & $10^{\Delta\cdot\ntrpn}$ \\ & \\
        $\trpnf$ & $\dfrac{10^{\Delta\cdot\ntrpn}}{\trpnf(10^{\Delta\cdot\ntrpn}-1)+1}$ \\ & \\
        $\ntrpn$ & $\dfrac{\log\left(\frac{\koff}{\kon}\frac{\trpnf}{1-\trpnf}\right)}{\Delta\cdot\ntrpn+\log\left(\frac{\koff}{\kon}\frac{\trpnf}{1-\trpnf}\right)}$ \\ & \\
        \bottomrule
    \end{tabularx}
    \caption{The mapping of sarcomere properties to F-pCa variations is non-unique. By scaling each parameter $p_i$ by the corresponding $\alpha_i$ coefficient, the cellular contraction model simulates a $\pCaf$ value which is shifted by exactly $\Delta$ units from the reference value.}
    \label{tab:alphavalues}
\end{table}

\begin{figure}[h!]
    \myfloatalign
    \includegraphics[width=0.8\textwidth]{figures/chapter08/Fig4.pdf}
    \caption{Different parameters can be individually perturbed to achieve the very same shift in the force-calcium relationship. Example showing $\SI{2}{\percent}$ leftwards (upper plot) and rightwards (bottom plot) shifts of the reference $\pCaf$ value. As $\trpnf$ and $\ntrpn$ also regulate the force-pCa curve's Hill coefficient (equation~\eqref{eq:h}), notice that for these two parameters the shift in the $\pCaf$ value is also affecting the curve's slope.}
    \label{fig:2perclwrwshift}
\end{figure}


%
%
%
\subsection{$\pCaf$ changes are non-uniquely explained by sarcomere alterations: the two-parameter case}\label{sec:changespCa50result2}
$\pCaf$ evaluation on $2$-parameter grids highlighted the presence of isolines, whose parameter points induce the same $\pCaf$ feature value (Figure~\ref{fig:pca50isolines}A). This means that moving between any two points on two isolines will result in the same shift in $\pCaf$, demonstrating also in the two-parameter case the non-uniqueness of changes in the half-activation of the F-pCa curve and a change in sarcomere properties.


%
%
%
\subsubsection{Non-unique mapping of F-pCa curve to LV function}\label{sec:fpcatolvnonuniquemapping}
Parameter points inducing the same shifts in the $\pCaf$ feature value (isolines of Figure~\ref{fig:pca50isolines}A) are quantitatively linked to different EF feature values (Figure~\ref{fig:pca50isolines}B), showing that the mapping from F-pCa curve to LV function is not unique.

\begin{figure}[h!]
    \myfloatalign
    \includegraphics[width=\textwidth]{figures/chapter08/Fig5.pdf}
    \caption{For each pair of parameters regulating the $\pCaf$ feature of the F-pCa curve, a $2$D uniform grid is constructed using a $\pm\SI{50}{\percent}$ perturbation around the reference parameter set values (black dots). (A) The cell contraction model is then used to calculate the $\pCaf$ feature value at every parameter point of the grid, and each grid is plotted as a heat map with values given as percentages from the control $\pCaf$ value. Contour plots are finally added to highlight the presence of isolines, whose many different parameter sets induce the same shift ($-\SI{2}{\percent}$ dotted, $\SI{0}{\percent}$ full, $+\SI{2}{\percent}$ dashed black lines) in the control $\pCaf$ value. (B) Parameter points from the obtained isolines are mapped by the emulator into EF feature values. These values (given as percentages of the control EF value) are represented by different colour intensities used to colour each parameter point in the isolines, showing that parameters that share the same $\pCaf$ values are mapped to different EF values.}
    \label{fig:pca50isolines}
\end{figure}


%
%
%
\subsection{LV function changes are non-uniquely explained by sarcomere alterations}\label{sec:changesLVfunctionresult}
EF evaluation on $2$-parameter grids highlighted the presence of isolines, where all parameter combinations along these lines result in the same predicted EF (Figure~\ref{fig:EFisolines}A). This means that moving between any two points on each isoline will result in the same EF, demonstrating the non-uniqueness of the LV function and sarcomere properties.


%
%
%
\subsubsection{Non-unique mapping of LV function to F-pCa curve}\label{sec:lvtofpcanonuniquemapping}
Parameter points that give rise to the same predicted EF value (isolines of Figure~\ref{fig:EFisolines}A) are quantitatively linked to different $\pCaf$ values (Figure~\ref{fig:EFisolines}B), showing that the mapping from LV function to the F-pCa curve is not unique.

\begin{figure}[h!]
    \myfloatalign
    \includegraphics[width=\textwidth]{figures/chapter08/Fig6.pdf}
    \caption{For each pair of parameters regulating the $\pCaf$ feature of the F-pCa curve, a $2$D uniform grid is constructed using a $\pm\SI{50}{\percent}$ perturbation around the reference parameter set values (black dots). (A) A trained emulator is then used to predict the $\textrm{EF}$ feature value at every parameter point of the grid, and each grid is plotted as a heat map with values given as percentages from the control $\textrm{EF}$ value. Contour plots are finally added to highlight the presence of isolines (full black lines), whose many different parameter sets induce the same $\SI{0}{\percent}$ shift in the control $\textrm{EF}$ value. (B) Parameter points from the obtained isolines are mapped by the cell contraction model into $\pCaf$ feature values. These values (given as percentages of the control $\pCaf$ value) are represented by different colour intensities used to colour each parameter point in the isolines, showing that parameters that share the same $\textrm{EF}$ values are mapped to different $\pCaf$ values.}
    \label{fig:EFisolines}
\end{figure}


%
%
%
\subsection{Non-unique mapping from/to F-pCa curve to/from LV function}\label{sec:ch8nonuniquemappings}
In Sections~\ref{sec:changespCa50result1}--\ref{sec:changespCa50result2}, we proved that changes in the $\pCaf$ value are not uniquely caused by changes in sarcomere properties. In Section~\ref{sec:fpcatolvnonuniquemapping}, we also showed that changes in the EF value are not uniquely caused by changes in the $\pCaf$ value. Because of these two findings, we can state that observed changes in the F-pCa curve cannot be uniquely mapped to changes in the LV function.

\vspace{0.2cm}
At the same time, in Section~\ref{sec:changesLVfunctionresult} we proved that changes in the EF feature value are not uniquely caused by changes in sarcomere properties. In Section~\ref{sec:lvtofpcanonuniquemapping}, we also showed that changes in the $\pCaf$ feature value are not uniquely caused by changes in the $\textrm{EF}$ value. Because of these two other findings, we can state that observed changes in the LV function cannot be uniquely mapped to changes in the F-pCa curve.


%
%
%
\section{Discussion}\label{sec:ch8discussion}
In this study, we have made use of mathematical models to characterise the relationship between sarcomere properties and the LV contractile function in the healthy rat heart. We have highlighted the presence of complex nonlinearities; in particular, we have demonstrated that the relationship between myofilaments' features (e.g. $\Caif$) and PV loop characteristics (e.g. EF) is non-monotonic. As these sarcomere properties also define the steady-state force-calcium relationship in the cardiac muscle, we extended this result in terms of shifts in the F-pCa curve that are often examined when experimentally assessing the effect of sarcomere-targeting pharmacological compounds. We have therefore provided both analytical and simulation study evidence that alterations in the F-pCa curve can cause very different changes in whole heart function. At the same time, we have shown that observed changes in the LV function cannot be attributed to a unique modification in sarcomere properties. Whereas in the previous sections we characterised the LV function using EF, the obtained results more generally hold true for many other clinically relevant indexes of LV both systolic and diastolic function, including IVRT, PeakP, maxdP, mindP etc. To design new treatment strategies, solely looking at the induced shift in the $\pCaf$ feature value of the F-pCa curve on its own is therefore not enough to predict or to understand predictions of how this will be translated into whole-organ function changes.

\vspace{0.2cm}
When studying myofilament $\Ca$ sensitivity using F-pCa curves, simply considering the $\Ca$ sensitivity as a stand-alone measurement is not sufficient for translation into dynamic contraction and relaxation. This concept was previously shown for single-cell dynamics by Chung et al.~\cite{Chung:2016} who highlighted the importance of biophysical measures of $\koff$ and $\kon$ to help predict tension dynamics at the cellular level. However, in this work, we expand this concept further to the whole heart level by analysing LV features and exploring the inverse mapping from LV features to F-pCa curves, highlighting the non-uniqueness of both forward and inverse mapping between F-pCa and LV features. We emphasise that, for instance, an increase in $\Ca$ sensitivity, i.e. a leftward shift in the F-pCa curve, will have two effects. First, it will increase residual tension, decreasing EDV; second, it will increase end-systolic tension, decreasing ESV. The balance of these two effects will impact the change in SV and EF for a given change in $\Ca$ sensitivity. This impact of changes in $\pCaf$ depends on the starting F-pCa curve, the $\Ca$ transient, cardiac material properties and boundary conditions. Determining these multi-scale relationships from experimental preparations remains challenging.

\vspace{0.2cm}
The performed GSA interpreted the variability of EDV, ESV, SV and EF in terms of the uncertainty in F-pCa curve-modulating sarcomere parameters, and it emphasised the importance of the $\Ca$ sensitivity into affecting these PV loop characteristics. A deeper insight can be gained if we look at higher-order interactions' effects. Although these are present for all the four LV features considered, they are remarkably high only for the SV and EF features, explaining more than half of the total variance for SV and almost half of the total variance for EF. We can notice that for the EDV and ESV features higher-order interactions' effects are instead small, and dominating effects are mostly of the first-order type. Conversely, for the SV and EF features where higher order interactions' effects are high, dominating lower-order effects are mostly of the second-order type. All these considerations can be summarised by stating that although it is possible to interpret changes in terms of the individual contribution of parameters for the EDV and ESV features, this is not the case for SV and EF features. In this sense, it is more the joint effect of the whole sarcomere to determine the LV function rather than single myofilament components.

\vspace{0.2cm}
Indications from the GSA coupled with the information of F-pCa curve vs. LV function non-monotonicity shed a new light on the problem of interpreting pharmacological interventions' effects on the F-pCa curve in terms of the desired effects on whole-heart function. This concept is illustrated via the schematic in Figure~\ref{fig:schematic}, which highlights existing feedback mechanisms that regulate contraction in the heart. Modulation of the $\Ca$ transient can directly affect the active tension which is generated within the sarcomere, and alteration of the sarcomere generated force will eventually affect the PV loop. As the force-volume relationships at the end-diastolic and end-systolic pressures are fixed, sarcomere interventions that aim at shifting the F-pCa might result in no change in LV contractile function. For this reason, treatment strategies should aim at altering both the end-systolic and end-diastolic force-volume relationships or altering one while maintaining the other to yield an effect on EF.

\begin{figure}[ht!]
    \myfloatalign
    \includegraphics[width=\textwidth]{figures/chapter08/Fig7.pdf}
    \caption{Force-LV volume curves at the end-diastolic and end-systolic pressures (top right, orange and green lines) can be separately manipulated to modulate the LV contractile function. However, interventions on calcium transient (bottom left, blue line) and sarcomeric generated force (top left, orange and green full lines) both reflect into modifications of the former curves, which in turns causes modification of the PV loop (bottom right, blue full line). In addition, pharmacological modulations on the sarcomere might cause a shift of the force-calcium relationship (top left, orange and green dashed lines). However, this will preserve the EF (bottom right, blue dashed line) without improving LV contractile function, which was the desired outcome of the performed sarcomeric intervention.}
    \label{fig:schematic}
\end{figure}


%
%
%
\subsection{Limitations}\label{sec:ch8limitations}
The limitations of this study mainly concern the adopted modelling/emulation framework, and are presented fully in Chapter~\ref{cha:chapter9}, Section~\ref{sec:ch9limitations}.

%
%
%
\section{Summary}\label{sec:ch8summary}
We have used a biophysically detailed mathematical model of a healthy rat heart contraction to quantitatively map sarcomere properties to whole heart function. Using this mapping, we demonstrated that the relationship between F-pCa curve and LV function is non-linear and non-monotonic. This results in the non-interpretability of observed changes in the LV function in terms of unique sarcomere modulations, which highlights the need for muscle experimental findings to be put into a broader context where not only the $\pCaf$ and Hill coefficient, but also active and passive force, length and velocity dependencies, calcium transient and boundary conditions are analysed.