%==============================================================================
\chapter{Novel biophysical insight: in silico quantitatively mapping force-pCa curves to the whole heart contraction and relaxation}\label{cha:chapter8}
%==============================================================================
%
%
%
\begin{remark}{Outline}
    In this chapter, we investigate on the force-calcium relationship (F-pCa), as described by the cell contraction sub-model of our full personalised SHAM rat hart contraction model. Using a combination of simulator and emulators, we show that changes in the F-pCa cannot be described by unique changes in sarcomere properties. At the same time, we show that changes in the LV function (e.g. ejection fraction) cannot be described by unique changes in the sarcomere properties. By coupling these two information, we prove that the mapping from the F-pCa curve to the LV function and the correspoding inverse mapping are non-unique. This finding sheds a new light on the problem of interpreting pharmacological interventions' effects on the F-pCa curve in terms of the desired effects on whole-heart contraction and relaxation.
\end{remark}



% \begin{figure}[!ht]
%     \myfloatalign
%     \includegraphics[width=\textwidth]{figures/chapter08/1param_pCa50_same_shift.pdf}
%     \caption{Different parameters can be individually perturbed to achieve the very same shift in the force-calcium relationship. However, the performed parameter perturbation will result in different twitch transients and corresponding different pressure-volume loops. Example showing $\SI{2}{\percent}$ leftwards shift of the $\pCaf$ value.}
%     \label{fig:oneparamsameshifttab}
% \end{figure}


% \begin{figure}[!ht]
%     \myfloatalign
%     \includegraphics[width=\textwidth]{figures/chapter08/same_shift_FpCa_1param_different_Ts_and_PVloops.pdf}
%     \caption{Force-pCa curves non-unique mapping. With a single-parameter \textit{ad hoc} perturbation we can achieve the exact same shift in the $\pCaf$ (although in two cases resulting in different Hill coefficients). However, the same parameter perturbations lead to very different generated force, which in turn produce different PV-loops at the whole-organ scale. Each colour represents the perturbation in the indicated parameter.}
%     \label{fig:oneparamsameshift}
% \end{figure}


% \begin{figure}[!ht]
%     \myfloatalign
%     \includegraphics[width=\textwidth]{figures/chapter08/LVfeat_to_FpCa_schematic.pdf}
%     \caption{LV features non-unique mapping to force-pCa curves. (1) A uniform grid of equally spaced parameters $p_1$ and $p_2$ values is mapped by the emulator into predictions of feature $y$ values. (2) An entire isoline of parameter sets $(p_1,p_2)$ sharing the same unchanged $y$ feature from reference value exists. (3) Parameters sets are extracted from the isoline using a threshold of $10^{-4}$. (4) The cell contraction model is run at each of these parameter sets to generate steady-state force-pCa curves.}
%     \label{fig:gridmappingschematic}
% \end{figure}


% % \begin{figure}[ht!]
% %     \myfloatalign
% %     \subfloat[EF feature]
% %     {\label{fig:twoparamssameef}
% %     \includegraphics[width=\linewidth]{figures/chapter08/Ca50_koff_vs_EF.png}}\quad
% %     \subfloat[IVRT feature]
% %     {\label{fig:twoparamssameivrt}
% %     \includegraphics[width=\linewidth]{figures/chapter08/Ca50_koff_vs_IVRT.png}}
% %     \caption{Different Force-pCa curves are mapped to the same LV feature. A $2$-parameter grid is mapped using a trained GPE into the corresponding LV feature value for each pair of parameters within the grid. The same pairs of parameters are then used to simulate steady-state force-calcium relationships. Example showing ($\Caif,\,\koff$) parameters variations mapped into EF and IVRT features. The control parameter set is displayed as a red dot.}\label{fig:twoparamssamefeatdifffpca}
% % \end{figure}



% \begin{figure}[ht!]
%     \myfloatalign
%     \subfloat[Same $\pCaf$ mapping to different EF.]
%     {\label{fig:twoparamssameef}
%     \includegraphics[width=\linewidth]{figures/chapter08/FpCa_space_mapping_Ca50_koff_vs_EF.png}}\quad
%     \subfloat[Same EF mapping to different $\pCaf$.]
%     {\label{fig:twoparamssameivrt}
%     \includegraphics[width=\linewidth]{figures/chapter08/LVfeats_space_mapping_Ca50_koff_vs_EF.png}}
%     \caption{FpCa non-unique mapping to LV pump function. A $2$-parameter grid is mapped using (a) the cell contraction model or (b) a trained GPE into (a) $\pCaf$ values or (b) EF feature values for each pair of parameters within the grid. Pairs of parameters belonging to the $\SI{100}{\percent}$ isoline are then used to (a) predict EF feature values or (b) simulate $\pCaf$ values. Example showing ($\Caif,\,\koff$) parameters variations mapping. The reference parameter set is displayed as a black dot.}\label{fig:twoparamssamefeatdifffpca}
% \end{figure}

