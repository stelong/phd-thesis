%==============================================================================
\chapter{Summary and concluding remarks}\label{cha:chapter9}
%==============================================================================
%
%
%
\section{Contribution}\label{sec:ch9contribution}
The main contribution of this thesis is to have shown the feasibility of incorporating pre-existing probabilistic tools into pre-existing complex deterministic cardiac models, thus improving our understanding of the models themselves and enhancing their capabilities in terms of real world applications. The developed framework can be easily extended/adapted to model differently complex/different biological systems. Specifically for the cardiac modelling community, this thesis constitutes an important step towards application to the more complex human heart for personalised medicine. \todo{example applications of this framework can already be found in... cite co-author papers maybe? cristobal (anatomy/shape), adelisa (cell EP), damiano (tissue EP), elias (haemodynamics)...}


%
%
%
\section{Limitations}\label{sec:ch9limitations}
This thesis work has got many limitations which can be grouped under five main topics: rat heart model, experimental data, model emulation, emulation-based model fitting and global sensitivity analysis. We shall discuss each of these separately in the next paragraphs.


%
%
%
\subsection{Rat heart model}\label{sec:ch9rat_heart_model}
We have developed a $3$D biophysically-detailed model of rat heart contraction mechanics. This model is inherently a simplification of the underlying real biological system it aims to \textit{in silico} represent.

\vspace{0.2cm}
Firstly, the model is a two-chamber simplification of a real heart (no atria), and spatial boundary conditions do not account for the pericardium. Furthermore, this is not a closed loop system, being pressure boundary conditions modelled using a three-element Windkessel model. Because of the used boundary conditions, this virtual rat heart is closer to an \textit{ex vivo} preparation rather than to an \textit{in vivo} heart. All these factors (missing atria and pericardium, no closed loop system) could play a role in constraining the cardiac mechanics~\cite{Strocchi:2020}\todo{cite paper on atria influence on mechanics}.

\vspace{0.2cm}
We have also not accounted for potential spatial variations in cellular properties and $\Ca$ transient, and the latter homogeneously activates contraction throughout ventricular walls. For what concerns single cell contractile function, both thin and thick filament dynamics were modelled using a simplified representation of the sarcomere, without a detailed mechanistic description of its components. Specifically, the cross-bridge kinetics is described by a two-state model where the strongly-/weakly-/un-bound states are collapsed into a single state. While more detailed models exist~\cite{Land:2015}, their parameters are not necessarily constrained using experimental data due to difficulties in measuring subcellular processes, and are not easily integrable into multi-scale whole-organ simulations. The used model is still able to recapitulate all the main sarcomere processes including the length and velocity dependencies.


%
%
%
\subsection{Experimental data}\label{sec:ch9experimental_data}

%
%
%
\subsection{Model emulation, fitting and uncertainty quantification}\label{sec:ch9model_emulation_fitting_and_uncertainty_quantification}
We fitted the model to the data using an emulator, which itself has uncertainty, and this can increase the overall uncertainty. A self-consistent multi-scale data of the effects of OM on cell, tissue and organ scales would allow us to better identify and address the source of these discrepancies.

\vspace{0.2cm}
If the model is an approximation of the real system it represents, when we substitute it with an emulator we are adding an extra level of model discrepancy which will require further experts knowledge to be quantified.

\vspace{0.2cm}
For both HM and GSA we used independent GPEs for each output. This univariate approach provided us the flexibility to tune the hyperparameters and choose basis functions for each output independently. However, it did not account for potential correlations in outputs, which could be accounted for with a multivariate strategy e.g.~\cite{Conti:2009}, although this assumes common hyperparameters across all outputs. A multi-variate implausibility measure could be constructed from multi-output GPEs, which could potentially assist in ruling out more implausible parameter space.

\vspace{0.2cm}
Third, the use of emulators adds an additional layer of uncertainty in modelling the LV function, since emulators are probabilistic surrogates of the simulator, which is already an \textit{in-silico} representation of the real world biological system. However, the speed-up that we gain by using emulators which enables efficient identification of the sarcomere parameters which play a key role in determining the LV function at whole-organ overcomes by far the slight loss in quantitative accuracy when making predictions.

\todo{start Model fitting}
Secondly, HM provides a bounded region of non-implausible parameter sets. Parameter bounds do not define parameter distributions. Including a Markov chain Monte Carlo parameter fit using the HM bounds as priors would extend this method to estimate likely parameter distributions as opposed to parameter bounds.

\vspace{0.2cm}
The heterogeneity of the data used to constrain the models is another limitation. Specifically, pressure measurements were not available for the rats used as an animal model and were therefore collected and averaged over similar experimental studies. Related to this, we only used a single representative anatomy for the healthy rats' cohort and single representative anatomy for the aortic-banded rats' cohort: ET and IVRT features were constrained according to single measurements coming from these two anatomies' segmentations.

\vspace{0.2cm}
In the implausibility measure (equation ($4$) in the Supplementary materials) calculation we omitted the model discrepancy term, for which we did not have any estimate available. To completely replace animal models with virtual representations of them, further experts knowledge will be needed to quantify the difference between the \textit{in-silico} model and the real world system that it represents.


%
%
%
\section{Next steps}\label{sec:ch9next_steps}
The next steps will go towards improving the methodology this thesis is heavily based on.

\vspace{0.2cm}
The first matter we want to address is the optimal collection of training dataset points via simulator evaluations. As more complex simulators will require more computational resources, or when predictions will have to be derived almost in real time scenarios, we will not be able to afford simulating a whole parameter sweep (sampled with a space-filling design), discard failed simulations, and keep converging simulation to finally form the emulator training dataset. Space-filling designs assume that the samples provide information equally across the entire input space, thus they perform exploration of the whole input space~\cite{Yue:2021}. However, these designs are not adaptive to the information from the response surface (input-output relationship). Identifying the nature of input-output relationships to subsequently choose design points one at a time e.g. in order to maximise the expected improvement on a given objective function realises a trade-off between exploration and exploitation, and is already possible through different \textit{acquisition functions} available in the literature for \textit{active learning} tasks (e.g.~\cite{Jones:1998,Pasolli:2011,Schreiter:2015}). We aim at systematically incorporate an active learning component in our emulation framework to drastically reduce the number of simulator evaluations needed to build emulators for accurate regression.

\vspace{0.2cm}
A second topic of further investigation is the emulator structure we adopted. Specifically, we chose to use univariate emulators to map model input parameters to each of the LV scalar features independently. In future studies, we aim at investigating whether a multivariate approach can provide better model predictions. Building a multivariate GPE requires prescribing the covariance structure between the output components~\cite{Bonilla:2008,Rougier:2008,Conti:2010}, and it is not a trivial task. However, the output features we considered were extracted from two specific curves, namely the LVV and the LVP. This fact can be used to infer the correlations between output features belonging to the same curve. For example, in the case of the timing magnitudes ET, IVCT, IVRT, Tdiast$-$IVRT we already know that their sum cannot exceed the cardiac cycle length. This information, along with other physiologically-derived arguments on LV volume and pressure transient morphologies, could be potentially used to prescribe the multivariate GPE covariance structure. Related to this, a multivariate implausibility measure could be constructed from multivariate GPEs, and potentially assist in ruling out more implausible parameter space.

\vspace{0.2cm}
The third improvement we have in mind concerns the variance-based sensitivity analysis. We have seen that when using emulators to estimate the Sobol' sensitivity indices, the numerical error coming from having estimated expectations' and variances' integrals using quadrature formulae was not taken into account, but capturing the emulator uncertainty in the estimates was prioritised instead. 