%==============================================================================
\chapter{In silico mapping of sarcomere pharmacological modulations to whole-organ function and back again: the omecamtiv mecarbil case study}\label{cha:chapter5}
%==============================================================================
%
%
%
\begin{remark}{Outline}
    In this chapter, we use the personalised SHAM rat heart contraction model as a reference model of a healthy rat heart to show that it is possible to quantitatively map pharmacological interventions at the sarcomere level through to the whole-organ function. Moreover, we show that we can infer cell-level pharmacological effects from whole-organ observations, demonstrating the feasibility of the inverse mapping. We apply this framework to an example drug, namely omecamtiv mecarbil, which has been proposed as a possible treatment of HF.
    We first present the variability observed in the drug effects from preclinical data, namely from ($1$) \textit{in vivo} whole-heart haemodynamics measurements and ($2$) \textit{in vitro} steady-state F-pCa measurements (Section~\ref{sec:ch5preclinical_data}). We then model the drug mechanism of action using $4$ model parameters (Section~\ref{sec:ch5inputparameterspace}), and we use emulators to map these to the LV features and to characterise the latter global sensitivities (Section~\ref{sec:ch5trainingdatasetandemulators}). We then constrain the $4$-parameter space using the preclinical data (Sections~\ref{sec:ch5inferringomeffectswholeorganfpca}--\ref{sec:ch5inferringomeffectsfpcawholeorgan}). The constrained parameter space from case ($1$) is then mapped to F-pCa curves (Section~\ref{sec:ch5model_predicted_om_effects_on_the_fpca_relationship}), while the one resulting from case ($2$) is mapped to the LV features (Section~\ref{sec:ch5model_predicted_om_effects_on_the_lv_function}). We then include a discussion of the results and address specific limitations (Section~\ref{sec:ch5discussion}), and we conclude with a brief summary (Section~\ref{sec:ch5summary}).
\end{remark}


%
%
%
\section{Motivation}\label{sec:ch5motivation}
HF is a leading cause of hospitalisation worldwide, with more than one million admissions annually in the US and Europe~\cite{Benjamin:2018}. However, the treatment options are limited, and, therefore, new pharmacotherapies are continuously sought. HF pathways can involve impaired cellular function and propagate up to the whole-organ dysfunction. Multi-scale contraction modelling represents a valuable tool for understanding the underlying mechanisms and possibly identifying targets in these pathways.

\vspace{0.2cm}
Building on the developed rat heart contraction modelling framework (Chapter~\ref{cha:chapter4}), we can investigate mechanisms of drug action and better understand the mechanistic processes linking cellular and whole-heart contraction. For this case study, we chose omecamtiv mecarbil (\acs{OM}), a novel drug currently in Phase $3$ clinical trial~\cite{Teerlink:2021} for treating HF. OM is a selective allosteric cardiac myosin modulator. It increases the rate of cross-bridge cycling by accelerating phosphate release~\cite{Malik:2011} without disrupting intracellular calcium dynamics~\cite{Horvath:2017}. Recently, OM was shown to enhance the duty ratio, resulting in increased calcium sensitivity and slowed force development~\cite{Swenson:2017, Kampourakis:2018}.

\vspace{0.2cm}
This chapter outlines our methodology of incorporating OM into our model, which consists of calibrating the cellular model using \textit{in vitro} data from skinned cellular and trabecular preparations in OM-containing solutions~\cite{Nagy:2015,Kampourakis:2018,Kieu:2019}, and validating the biventricular multi-scale contraction model using pressure-volume measurements from \textit{in vivo} whole-heart studies in healthy animals with OM~\cite{Bakkehaug:2015}. Our simulation results are consistent with the available experimental data on OM and, importantly, support the hypothesis (e.g.~\cite{Swenson:2017}) that OM affects the thin filament (in addition to the expected effect on thick filaments due to OM being the selective cardiac myosin activator).

\vspace{0.2cm}
This work, incorporating our previous publication~\cite{Longobardi:2021}, demonstrates how quantitative mapping from cellular to whole-organ level can be used to improve our understanding of drug action mechanisms.


%
%
%
\section{Preclinical data}\label{sec:ch5preclinical_data}


%
%
%
\subsection{Cell-level measurements}\label{sec:ch5celllevelmeasurements}
\textit{In vitro} measurements of the OM effects on the sarcomere in healthy rats were taken from literature experimental studies performed on skinned myocytes' preparations. OM effects were given in terms of alterations of the $\pCaf$ and $h$ features of the F-pCa curve (introduced in Section~\ref{sec:ch2theforcecalciumrelationship}, equation~\eqref{eq:FpCa}). To map these effects from skinned to intact/\textit{in vivo} results, we used percentages of F-pCa features' change from the control muscle to the OM-containing solution-exposed muscle. These values are summarised in Table~\ref{tab:pshiftpslope}.

\begin{table}[ht!]
    \myfloatalign
    \begin{tabularx}{\textwidth}{lXX}
    \toprule
    \tableheadline{Fraction of change} & \tableheadline{Exp. variability} & \tableheadline{Reference}  \\
    \midrule
    $P_{\textrm{shift}}$ (for $\pCaf$) & $[0.0085,\,0.0833]$ & \cite{Nagy:2015, Kampourakis:2018, Kieu:2019} \\
    $P_{\textrm{slope}}$ (for $h$) & $[-0.7282,\,-0.2470]$ & \cite{Nagy:2015, Kampourakis:2018, Kieu:2019} \\
    \bottomrule
    \end{tabularx}
    \caption{F-pCa curve features' fractions of change experimental variability in OM-containing solution-exposed healthy rat skinned muscles. Values are given as ranges of minimum and maximum fractions of change from control experimental mean values.}
    \label{tab:pshiftpslope}
\end{table}



%
%
%
\subsection{Whole-organ measurements}\label{sec:ch5wholeorganlevelmeasurements}
We did not have \textit{in vivo} measurements of the OM effects on the whole-organ function in healthy rats, so qualitative observations from a healthy pig study were used~\cite{Bakkehaug:2015}. Again, we used percentage changes in LV features from baseline values after OM administration. In particular, we focused on the features whose change from control was reported to be statistically significant. These values are summarised in Table~\ref{tab:pigdata}.

\begin{table}[ht!]
    \myfloatalign
    \begin{tabularx}{\textwidth}{lXX}
        \toprule
        \tableheadline{LV feature} & \tableheadline{Exp. variability ($\SI{}{\percent}$)} & \tableheadline{Reference} \\
        \midrule
        $\textrm{EDV}^*$ & $87.14 \pm 17.14$ & \cite{Bakkehaug:2015} \\
        $\textrm{ESV}^*$ & $76.92 \pm 20.51$ & \cite{Bakkehaug:2015} \\
        $\textrm{SV}$ & $100.00 \pm 25.81$ & \cite{Bakkehaug:2015} \\
        $\textrm{EF}$ & $115.91 \pm 18.18$ & \cite{Bakkehaug:2015} \\
        $\textrm{ET}^*$ & $116.16 \pm 9.61$ & \cite{Bakkehaug:2015} \\
        $\textrm{Tdiast}^*$ & $88.24 \pm 16.97$ & \cite{Bakkehaug:2015} \\
        $\textrm{maxdP}^*$ & $121.66 \pm 35.53$ & \cite{Bakkehaug:2015} \\
        $\textrm{mindP}$ & $108.96 \pm 31.01$ & \cite{Bakkehaug:2015} \\
        \bottomrule
    \end{tabularx}
    \caption{Left ventricular features' experimental variability in healthy pigs after OM administration. Values are given as mean$\pm$std percentage change from control experimental mean values. Asterisked $(\cdot)^*$ features' changes were reported to be statistically significant.}
    \label{tab:pigdata}
\end{table}


%
%
%
\section{Methods}\label{sec:ch5methods}


%
%
%
\subsection{Rat heart contraction model}\label{sec:ch5ratheartcontractionmodel}
To quantitatively link sarcomere properties to whole-organ function, we used the personalised SHAM rat heart contraction model derived in Section~\ref{sec:ch4fittedmodels}. Again, this model (or simulator) can be seen as a multi-scale map from input parameters to output features.

\vspace{0.2cm}
With the simulator input, we aimed at modelling the OM effect at the sarcomere level.
As OM increases the rate of cross-bridge formation~\cite{Malik:2011}, we considered parameters that are specifically responsible for cross-bridge dynamics in the Land et al.~\cite{Land:2012} cell contraction sub-model, namely $\kxb$, $\nxb$, $\trpnf$ and $\tref$. These parameters, introduced in Section~\ref{sec:ch2contractionmodel} and again reported in Table~\ref{tab:paramswithdefom}, constituted the simulator $4$-dimensional input.

\begin{table}[ht!]
    \myfloatalign
    \begin{tabularx}{\textwidth}{llX}
    \toprule
    \tableheadline{Parameter} & \tableheadline{Units}                   & \tableheadline{Definition} \\
    \midrule
    $\kxb$                    & $\SI{}{\per\milli\second}$              & cross-bridges cycling rate \\
    $\nxb$                    & $-$                                     & cross-bridge formation degree of cooperativity \\
    $\trpnf$                  & $-$                                     & fraction of $\Ca$-TnC bounds for half-maximal cross-bridges activation \\
    $\tref$                   & $\SI{}{\kilo\pascal}$                   & maximal reference tension \\
    \bottomrule
    \end{tabularx}
    \caption{Model parameters and their definitions.}
    \label{tab:paramswithdefom}
\end{table}

\vspace{0.2cm}
The simulator output described the LV function using a set of scalar features. These were the same $12$ features used in the previous analysis (Table~\ref{tab:lvfeatures}) with the addition of other $2$ features (Table~\ref{tab:lvfeaturesom}), for a total of $14$ features.

\begin{table}[ht!]
    \myfloatalign
    \begin{tabularx}{\textwidth}{llX}
    \toprule
    \tableheadline{LV feature}                  & \tableheadline{Units}                         & \tableheadline{Definition} \\ \midrule
    $\textrm{SV}$                  & $\SI{}{\micro\liter}$                  & stroke volume \\
    $\textrm{ET/Tdiast}$                  & $-$                  & systolic ejection time over diastolic filling time \\
    \bottomrule
    \end{tabularx}
    \caption{Additional LV features of interest.}
    \label{tab:lvfeaturesom}
\end{table}

\noindent
The obtained simulator map had thus the following form:
%
\begin{align}\label{eq:fsimulom}
    f_{simul}\colon\mathbb{R}^{4} &\to\underbrace{\mathbb{R}\times\cdots\times\mathbb{R}}_{14\,\text{times}} \\
    \mathbf{x} &\mapsto (y_1,\,\dots,\,y_{14}) \nonumber
\end{align}

\noindent
When running the simulator at a new parameter point, we fixed all the parameters not included in the set of simulator inputs to the personalised SHAM rat model baseline parameter values (Table~\ref{tab:shamabbestfitparamvalues}) when applicable, or to the Land et al.~\cite{Land:2012} model baseline values when otherwise.


%
%
%
\subsection{Input parameter space}\label{sec:ch5inputparameterspace}
The input parameter space $X\subset\mathbb{R}^{4}$ of the simulator map $f_{simul}$ introduced in equation~\eqref{eq:fsimulom} was defined as the hypercube obtained by the Cartesian product of $4$ one-dimensional parameter ranges. Each of these ranges was given as the combination of bounds inferred from the \textit{in vitro} F-pCa data presented in Section~\ref{sec:ch5celllevelmeasurements} (further details are provided in Section~\ref{sec:ch5inferringomeffectswholeorganfpca}) and a $\pm\SI{50}{\percent}$ ($\pm\SI{30}{\percent}$ for $\tref$) perturbation around the related parameter reference value. Adopted ranges are reported in Table~\ref{tab:omparamranges}.

\begin{table}[ht!]
    \myfloatalign
    \begin{tabularx}{\textwidth}{XXX}
        \toprule
        \tableheadline{Parameter} & \tableheadline{Units} & \tableheadline{Range} \\
        \midrule       
        $\kxb$   & $\SI{}{\per\milli\second}$ & $[0.0086,\,0.0258]$ \\
        $\nxb$   & $-$ & $[0.90,\,7.05]$ \\
        $\trpnf$ & $-$ & $[0.05,\,0.50]$ \\
        $\tref$  & $\SI{}{\kilo\pascal}$ & $[109.25,\,202.89]$ \\
        \bottomrule
    \end{tabularx}
    \caption{Parameters' ranges used for describing the healthy rat model $4$D input parameter space.}
    \label{tab:omparamranges}
\end{table}



%
%
%
\subsection{Training dataset, emulators and global sensitivity analysis}\label{sec:ch5trainingdatasetandemulators}
We sampled $4096$ points from a LHD over the input parameter space $X$ defined in Section~\ref{sec:ch5inputparameterspace}. The simulator was run at these points, and the successfully completed simulations were collected to form the training dataset ($1189$ points).

\vspace{0.2cm}
Univariate GPEs, defined as in Section~\ref{sec:ch3gaussianprocessemulation}, were used to predict each of the $14$ LV output features, and all the GPE model hyperparameters (both from the mean function and from the zero-mean GP) were jointly optimised during training by maximisation of the model log marginal likelihood (equation~\eqref{eq:logmarginallikelihood}). For each of the $14$ trained GPEs, we used the $R^2$-score to check the regression accuracy and the $ISE_2$ to assess the adequacy of being used as a surrogate model (Section~\ref{sec:ch3regressionaccuracy}). The GPE implementation and training were performed using \texttt{GPErks} emulation tool~\cite{GPErks:2021} based on \texttt{GPyTorch} Python library~\cite{Gardner:2019}.

\vspace{0.2cm}
To study the input parameters' impact on the output LV features' total variance, we performed a GSA using the trained GPEs. Model outputs' sensitivity to model inputs was characterised by Sobol' first-order and total effects. These were estimated using \texttt{SALib} Python library~\cite{Herman:2017}. \texttt{GPErks} tool~\cite{GPErks:2021} was used to incorporate GPEs' full posterior distribution samples to account for emulators' uncertainty in Sobol' indices' estimates, by following the second emulation-based approach of equations~\eqref{eq:emulpostsamplesgsa1}--\eqref{eq:emulpostsamplesgsa2}, presented in Section~\ref{sec:ch3emulatorbasedestimates}. Parameters whose Sobol' indices' distributions' expectation was below the threshold $0.01$ were determined to have negligible effects.


%
%
%
\subsection{Inferring OM effects on whole-organ function from in vitro F-pCa measurements}\label{sec:ch5inferringomeffectswholeorganfpca}
We wanted to investigate whether it is possible to validate OM mechanisms of action by using the built quantitative link between sarcomere properties and whole-organ function.

\vspace{0.2cm}
By recalling equations~\eqref{eq:ec50}--\eqref{eq:h}:
%
\begin{align}
    & \pCaf = -\log\left[\Caift\left(\frac{\koff}{\kon}\frac{\trpnf}{1-\trpnf}\right)^{1/\ntrpn}\right] \\
    & h = \nxb\ntrpn(1-\trpnf)
\end{align}

\noindent
we can notice that among the $4$ chosen input parameters (Section~\ref{sec:ch5ratheartcontractionmodel}) there are $2$ parameters, namely $\trpnf$ and $\nxb$, that can be used to tune both the $\pCaf$ and $h$ at the same time. Given a set of changes $(\Delta\pCaf,\,\Delta h)$ in these two features of the F-pCa curve, it is possible to scale the $2$ parameters' reference values such that the F-pCa curve undergoes a perturbation of exactly $(\Delta\pCaf,\,\Delta h)$. This can be accomplished by solving for the set of scaling coefficients $(\alpha,\,\beta)\in\mathbb{R}\times\mathbb{R}$ the following equations:
%
\begin{align}
    & \pCaf(\mathbf{p}_{\textrm{new}}) - \pCaf(\mathbf{p}) = \Delta\pCaf \label{eq:pcafeqalpha} \\
    & h(\mathbf{p}_{\textrm{new}}) - h(\mathbf{p}) = \Delta h \label{eq:heqbeta}
\end{align}

\vspace{0.2cm}\noindent
where $\pCaf(\cdot)$ and $h(\cdot)$ are now seen as functions of the reference $\mathbf{p}=(\trpnf,\,\nxb)$ and the new $\mathbf{p}_{\textrm{new}}=(\alpha\cdot\trpnf,\,\beta\cdot\nxb)$ parameter vectors.

\vspace{0.2cm}
The resulting $2$-dimensional space, given by all the possible $\mathbf{p}_{\textrm{new}}$ vectors obtained by solving equations~\ref{eq:pcafeqalpha}--\ref{eq:heqbeta} for all the viable $(\Delta\pCaf,\,\Delta h)$ according to the experimentally observed variability for the F-pCa curve (Table~\ref{tab:pshiftpslope}), will constitute an OM-compatible sarcomere space encoded by the $\trpnf$ and $\nxb$ parameters. To obtain the OM-compatible sarcomere space, we first generated a LHD of $100,000$ points over the $2$D space of F-pCa curve's features experimental variability (expressed as fractions of control/no-drug values) (Table~\ref{tab:pshiftpslope}). We then projected this space into the corresponding $2$D sarcomere parameter space as described above.

\vspace{0.2cm}
We finally mapped the obtained $2$D sarcomere parameter space to EDV, ESV, ET/Tdiast and maxdP using the corresponding GPEs, in order to see if these features were moving in the direction of change experimentally observed in the \textit{in vivo} pig haemodynamics after OM administration (Section~\ref{sec:ch5wholeorganlevelmeasurements}). The mapping was performed while keeping $\kxb$ and $\tref$ parameters, for which we had no data, fixed to their reference values.


%
%
%
\subsection{Inferring OM effects on the sarcomere from in vivo whole-organ measurements}\label{sec:ch5inferringomeffectsfpcawholeorgan}
We wanted to understand what the experimental \textit{in vivo} pig data of altered LV function with OM administration could tell us about the sarcomere parameter space of the model. To do this, we used a single iteration of the HM technique. This was done as described in Section~\ref{sec:ch3historymatching}, using the maximum implausibility measure
(equation~\eqref{eq:maximplmeasure}) across multiple target features, with the model discrepancy term (and its corresponding variance) set to zero in equation~\eqref{eq:implmeasure}. \texttt{Historia} tool~\cite{Historia:2021} was used to run the HM procedure.

\vspace{0.2cm}
For the target features, we aimed at matching the experimental variability observed in the LV features that showed a significant change from baseline after OM administration (asterisked features of Table~\ref{tab:pigdata}). At the same time, we wanted the model to preserve the ``control state'' for those features that did not show significant change from baseline. For this purpose, we matched for these features an experimental percentage variability of $\SI{100}{\percent}\pm\SI{0}{\percent}$ (i.e. no change). We sampled $400,000$ $NROY$ points from a LHD in the input parameter space $X$ (derived in Section~\ref{sec:ch5inputparameterspace}). As the GPEs had very high accuracy and very low variance for the predictions (shown in the results Section~\ref{sec:modelemulatorsandoutputsensitivities}), the HM single iteration was performed with half the common value for the implausibility cutoff, i.e. $I_{\,\textrm{cutoff}}=1.5$. This allowed to cut most of the space while still retaining few $X_{NIMP}$ points, as we shall see in Section~\ref{sec:ch5model_predicted_om_effects_on_the_fpca_relationship}.

\vspace{0.2cm}
The obtained $X_{NIMP}$ space constituted an OM-compatible sarcomere space encoded by the $\kxb$, $\nxb$, $\trpnf$ and $\tref$ parameters. The points from this space were finally mapped using the Land et al.~\cite{Land:2012} model of cellular contraction to F-pCa curves in order to see if the corresponding $\pCaf$ and $h$ features were moving in the direction of change experimentally observed in rats' skinned muscle preparations in OM-containing solutions (Section~\ref{sec:ch5celllevelmeasurements}).


%
%
%
\section{Results}\label{sec:ch5results}

%
%
%
\subsection{Model emulators and output sensitivities}\label{sec:modelemulatorsandoutputsensitivities}
All the trained GPEs had a cross-validation mean accuracy $>0.99$ and $>0.97$ for the $R^2$ score and $ISE_2$, respectively. These are reported in Table~\ref{tab:omgpesscores}. An example illustration of emulators making inference on a test set from the cross-validation process is provided in Figure~\ref{fig:omgpes}.

\begin{table}[ht!]
    \myfloatalign
    \begin{tabularx}{\textwidth}{XXX}
    \toprule
    \tableheadline{LV feature} & \tableheadline{$R^2$} & \tableheadline{$ISE_2 (\SI{}{\percent})$} \\
    \midrule
    $\textrm{EDV}$                 & $0.9967 \pm 0.0011$ & $99.07 \pm 0.61$ \\
    $\textrm{ESV}$                 & $0.9997 \pm 0.0000$ & $99.57 \pm 0.26$ \\
    $\textrm{EF}$                  & $0.9990 \pm 0.0002$ & $98.90 \pm 0.90$ \\
    $\textrm{IVCT}$                & $0.9932 \pm 0.0020$ & $98.23 \pm 0.72$ \\
    $\textrm{ET}$                  & $0.9955 \pm 0.0010$ & $99.15 \pm 0.59$ \\
    $\textrm{IVRT}$                & $0.9971 \pm 0.0009$ & $98.73 \pm 0.59$ \\
    $\textrm{Tdiast}$              & $0.9971 \pm 0.0005$ & $99.74 \pm 0.33$ \\
    $\textrm{PeakP}$               & $0.9991 \pm 0.0002$ & $97.14 \pm 0.85$ \\
    $\textrm{Tpeak}$               & $0.9952 \pm 0.0013$ & $98.90 \pm 0.77$ \\
    $\textrm{ESP}$                 & $0.9983 \pm 0.0003$ & $98.48 \pm 0.50$ \\
    $\textrm{maxdP}$               & $0.9959 \pm 0.0044$ & $99.49 \pm 0.61$ \\
    $\textrm{mindP}$               & $0.9961 \pm 0.0019$ & $99.24 \pm 0.31$ \\
    $\textrm{SV}$                  & $0.9991 \pm 0.0001$ & $98.99 \pm 0.77$ \\
    $\textrm{ET}/\textrm{Tdiast}$  & $0.9969 \pm 0.0006$ & $99.66 \pm 0.31$ \\
    \bottomrule
    \end{tabularx}
    \caption{GPEs' accuracy. The GPEs' accuracy was evaluated using the average $R^{2}$ score and $ISE_2$ obtained with a $5$-fold cross-validation. Values are reported as mean$\pm$std.}
    \label{tab:omgpesscores}
\end{table}

\begin{figure}[ht!]
    \myfloatalign
    \includegraphics[width=\textwidth]{figures/chapter05/bgpes_vs_bsplit_om.pdf}
    \caption{For each LV feature, the GPE with the highest $R^2$ split test score is used to make predictions at the respective left-out subset of test points. Predictions are sorted in ascending order for the sake of better visualisation and joined with a thick blue line, and the respective observations (empty dots) are sorted accordingly. $2$ STD confidence intervals (shaded regions) are also plotted around predicted mean lines.}
    \label{fig:omgpes}
\end{figure}

\vspace{0.2cm}
The performed GSA (Figure~\ref{fig:omgsa}) showed that $\trpnf$ was the most important parameter in explaining the LV features' total variance in the model. The second most important parameter was $\nxb$, followed by $\tref$ and $\kxb$ parameters. It is worth noticing that the two cross-bridge formation-regulating parameters that affect the LV function the most are also the ones that regulate the force-calcium relationship in the rat model.

\begin{figure}[ht!]
    \myfloatalign
    \includegraphics[width=\textwidth]{figures/chapter05/gsa_om.pdf}
    \caption{The impact of cross-bridge related sarcomere parameters on organ-scale LV features in the healthy rat. The contribution of each parameter is represented by its Sobol' main effect. For each LV feature, higher-order interactions (coloured in grey) are represented by the sum of all total effects minus the sum of all main effects.}
    \label{fig:omgsa}
\end{figure}



%
%
%
\subsection{Model predicted OM effects on the LV function}\label{sec:ch5model_predicted_om_effects_on_the_lv_function}
The OM-compatible $2$D sarcomere parameter space inferred from \textit{in vitro} rat F-pCa data is shown in Figure~\ref{fig:deltas}.

\begin{figure}[ht!]
    \myfloatalign
    \includegraphics[width=\textwidth]{figures/chapter05/Fig3_bonus.png}
    \caption{OM-compatible sarcomere space encoded by the $\nxb$ and $\trpnf$ parameters, inferred from \textit{in vitro} rat F-pCa data. A LHD of $100,000$ points in the ranges of experimental variability of the F-pCa curve (left panel) is projected into the corresponding plausible $2$D sarcomere parameter space (right panel). The reference parameter set is represented by a black dot, while reference dashed black lines divide the plane in four parts to highlight the quadrant of OM action.}
    \label{fig:deltas}
\end{figure}

\vspace{0.2cm}
Figure~\ref{fig:lvfeatsdistr} shows that the corresponding inferred median effects of these changes to $\pCaf$ and $h$ due to OM on whole-organ function show a qualitative agreement in the direction of change for all the $4$ LV features reported to be significantly altered by OM in the pig study~\cite{Bakkehaug:2015}.

\begin{figure}[ht!]
    \myfloatalign
    \includegraphics[width=\textwidth]{figures/chapter05/Fig3.pdf}
    \caption{Predicted \textit{in silico} OM effects on whole-heart function as percentage changes of LV features' values from control values. Experimental uncertainty ranges are displayed as shaded regions using percentage-from-control mean $\pm$ standard deviation values for both the healthy (grey) and $+$OM (orange) cases.}
    \label{fig:lvfeatsdistr}
\end{figure}


%
%
%
\subsection{Model predicted OM effects on the F-pCa relationship}\label{sec:ch5model_predicted_om_effects_on_the_fpca_relationship}
The HM first wave identified $3,469$ points (corresponding to $\SI{0.8672}{\percent}$ of the initial space) as non-implausible for replicating the organ-scale effects of OM administration. The resulting OM-compatible $4$D sarcomere parameter space inferred from \textit{in vivo} pig haemodynamic data is shown in Figure~\ref{fig:wave0}. 

\begin{figure}[ht!]
    \myfloatalign
    \includegraphics[width=\textwidth]{figures/chapter05/Fig1.png}
    \caption{OM-compatible sarcomere space encoded by the $\kxb$, $\nxb$, $\trpnf$ and $\tref$ parameters, inferred from \textit{in vivo} pig haemodynamic data. This space is obtained by running one iteration of HM which constrains the full parameter space according to an implausibility criterion that evaluates how plausible is a point to yield model predictions that are matching experimental observations.}
    \label{fig:wave0}
\end{figure}

\vspace{0.2cm}
Figure~\ref{fig:wave0mappingtofpCa} shows that the corresponding inferred median effects of these changes to the LV function due to OM on the intact F-pCa curve show a qualitative agreement in the direction of change for both the $\pCaf$ and $h$ features reported to be altered by OM in the rat skinned muscle studies~\cite{Nagy:2015, Kampourakis:2018, Kieu:2019}.

\begin{figure}[ht!]
    \myfloatalign
    \includegraphics[width=\textwidth]{figures/chapter05/Fig2.pdf}
    \caption{Predicted \textit{in silico} OM effects on the sarcomere as described by F-pCa curves calculated from the constrained, OM-compatible sarcomere parameter space and by the related percentage changes of $\pCaf$ and $h$ values from control values. Experimental uncertainty ranges are displayed (the middle and far right panels) as shaded regions using percentage-from-control mean $\pm$ standard deviation values for both the healthy (grey) and $+$OM (orange) cases.}
    \label{fig:wave0mappingtofpCa}
\end{figure}


%
%
%
\section{Discussion}\label{sec:ch5discussion}
We have demonstrated that the personalised, virtual rat heart framework (Section~\ref{sec:ch4fittedmodels}) can recapitulate the effects of OM on cardiac contraction and that multi-scale cardiac mechanics models can be used to infer the impact of drugs on cellular function from whole-organ observations. Calibration of model parameters to both cellular and whole-organ data on OM gives us additional insight into the mechanistic processes that link cellular and whole-heart contraction. Our results demonstrate that in order to reproduce available data, OM requires altering the function of both thick and thin filaments. The OM effect on thick filament (the direct site of action of OM) is essential to reproduce the OM effect on tension in whole-heart. Simulations show that the OM effect on F-pCa curves involves changes in thin filament function, namely by altering calcium myofilament sensitivity, which supports the recent hypothesis of the effect of OM on thin filaments~\cite{Swenson:2017}.


%
%
%
\subsection{Limitations}\label{sec:ch7limitations}
As a result of the adopted simplified approach for sarcomere contraction modelling (Section~\ref{sec:ch2contractionmodel}), the OM effect is modelled using $2$- to $4$-parameters linked to the cross-bridge cycling but not directly incorporating the OM mechanism of action (which is to increase the rate of myosin-head attachment to actin~\cite{Malik:2011}).

\vspace{0.2cm}
OM whole-organ data in healthy animals is limited. To the best of our knowledge, the \textit{in vivo} pig haemodynamics data~\cite{Bakkehaug:2015} used to constrain the model parameters is the only available non-human study of OM effect on LV function with therapeutic doses in healthy animals. Furthermore, differences across species and in acute vs chronic OM administration are not explored. For example, studies in healthy volunteers~\cite{Teerlink:2011} suggest an increase in SV and EF following OM treatment as a result of the overall systolic function improvement. However, we only mapped LV features' values that were reported to significantly change from control after OM administration~\cite{Bakkehaug:2015}, and our results excluded cases when a change in SV and EF features was observed. Additional studies are needed to investigate species differences and chronic effects of OM.

\vspace{0.2cm}
We have shown that the median model predictions qualitatively match the observed changes due to OM at the tissue and whole-organ scale. However, the model does not quantitatively match experimental observations. Specifically, there are some non-implausible Hill coefficients in Figure~\ref{fig:wave0mappingtofpCa} that are increased and do not qualitatively match the experimentally observed changes. Similarly, in some cases $\textrm{ET}/\textrm{Tdiast}$ is predicted to decrease in Figure~\ref{fig:lvfeatsdistr}, in contrast to the experimental observations. There are four potential contributors to this discrepancy. First, the tissue and organ drug effects are taken from different species. Second, the reference model parameters and anatomy are determined from a distinct set of experiments from the OM measurements. Third, the model is a simplification and may be missing some features. Forth, the experiments are performed in skinned preparations, while the model replicates intact tissue. This likely explains the discrepancy in the predicted change in the Hill coefficient, which is sensitive to the skinning process. A self-consistent multi-scale data of the effects of OM on cell, tissue and organ scales would allow us to identify better and address the source of these discrepancies.

\vspace{0.2cm}
More general limitations of the adopted modelling framework can be found in Chapter~\ref{cha:chapter9}, Section~\ref{sec:ch9limitations}.


%
%
%
\section{Summary}\label{sec:ch5summary}
We demonstrated how multi-scale heart contraction models can aid our understanding of the mechanistic links between cellular and whole-organ function and can help in interpreting skinned experimental preparations in the context of \textit{in vivo} whole-heart function. We have also shown that the same models can be used to map pharmacological effects on the sarcomere through to whole heart function and back again.