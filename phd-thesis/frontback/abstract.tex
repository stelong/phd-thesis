%*******************************************************
% Abstract
%*******************************************************
\pdfbookmark[1]{Abstract}{Abstract}
% \addcontentsline{toc}{chapter}{\tocEntry{Abstract}}

\begingroup

\let\clearpage\relax
\let\cleardoublepage\relax
\let\cleardoublepage\relax

\chapter*{Abstract}
Heart failure (HF) affects nearly a million people in the UK alone and increases the risk of cardiovascular diseases, stroke and death. At the whole-organ level, HF often manifests as impaired left ventricular (LV) contractile function. At the cellular level, LV contractile dysfunction is associated with altered sarcomere kinetics and disrupted calcium ($\Ca$) homeostasis. However, the link between cellular events and emerging pathological whole heart phenotypes is incompletely understood.

\vspace{0.2cm}
In this thesis, we aim to quantify the translation of cellular mechanisms to the LV contractile function and to elucidate the role of $\Ca$ and sarcomere dynamics in rat HF, with emphasis on the disease with preserved ejection fraction phenotype (HFpEF). We employed (Chapter~\ref{cha:chapter3}) a biophysically detailed 3D biventricular rat heart contraction mechanics model, which incorporated preload, afterload, fibre orientation, passive material properties, anatomy, $\Ca$ transient and sarcomere dynamics. The model cell-level function was described using a set of parameters (key regulators of the ionic processes and sarcomere contraction). The model organ-level behaviour was described using a set of features characterising tissue and haemodynamics properties, and the LV volume and pressure transients and the corresponding pressure-volume (PV) loop.

\vspace{0.2cm}
We first (Chapter~\ref{cha:chapter4}) fitted the model to real biventricular geometries, volumetric and functional data from a sham-operated (SHAM) and an aortic-banded (AB) 6-weeks-post-surgery rats, respectively representative of the healthy control and diseased rat cohorts from an experimental study on AB rats (diastolic HF animal model). We then characterised the LV features' sensitivity to model parameters. Model fitting was performed using the history matching (HM) technique, while uncertainty quantification was performed using Sobol' global sensitivity analysis (GSA). These normally require a large number of model evaluations to be performed. As the full forward model was too computationally expensive ($\sim 4-10$ hours per single forward calculation), we made HM and GSA performance computationally feasible by replacing the input-to-output multi-scale map with fast-evaluating ($\sim 1$ second per single forward calculation) probabilistic surrogates based on Gaussian process emulation (GPE). From now on, we will refer to the personalised (fitted) healthy SHAM rat model as ``the model''. The model constituted the starting point of the following three case studies.

\newpage
In the first case study (Chapter~\ref{cha:chapter5}), we used the model to show that it is possible to map pharmacological modulations from the sarcomere through to whole heart function and back again. As a case study, we validated the omecamtiv mecarbil (OM) mechanisms of action across scales in the healthy rat heart. Preclinical force-calcium (F-pCa) and LV haemodynamics data were used to constrain (using GPE $+$ HM) the parameter space to represent \textit{in silico} OM effects at the cellular level. The obtained spaces were then respectively mapped to the LV function and F-pCa curves to show that the model predictions are in qualitative agreement with the experimentally observed OM effects.

\vspace{0.2cm}
In the second case study (Chapter~\ref{cha:chapter6}), we first performed a validation against pharmacological channel blocking experimental literature data using a number of compounds by showing that the model can predict the observed effects on LV contractile function. Next (Chapter~\ref{cha:chapter7}), we used the model to generate a pathological model, representing the obese 20-week-old ZSF1 rat (HFpEF animal model). We then recovered the ZSF1 rat model back to the healthy state (using GPE $+$ GSA $+$ HM) by perturbing different sub-groups of parameters to represent different strategies of recovery to identify potential pharmacological, cellular targets for the treatment of HFpEF in rats.

\vspace{0.2cm}
In the third case study (Chapter~\ref{cha:chapter8}), we used the model to demonstrate that changes in the F-pCa relationship do not uniquely map to observed changes in the LV function and vice-versa. This result sheds new light on the assessment of myofilament calcium sensitivity using F-pCa shifts and the corresponding predictions on the LV contractile function based on F-pCa shifts.

\vspace{0.2cm}
We have built a virtual platform ($3$D rat heart contraction model $+$ GPE $+$ GSA $+$ HM techniques implementation codes available open access~\cite{zenodo:2021,Historia:2021,GPErks:2021}, scripts for generating figures available upon request) that can be used to efficiently test different pharmacological interventions and provide an indication of/identify potential cellular pharmaceutical targets for ``virtually'' treating HFpEF in rats. This was done by using computational models of cardiac mechanics and their probabilistic surrogates to quantify how normal/pathological cellular function is translated into normal/altered whole heart function. We have demonstrated the feasibility of applying Bayesian probabilistic techniques to small mammalian (rat) healthy and diseased $3$D models of cardiac mechanics. This thesis constitutes an important step towards applications to more complex systems ($3$D contractile human heart) for personalised medicine.

\vspace{0.2cm}


\endgroup

\vfill