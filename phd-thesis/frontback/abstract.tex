%*******************************************************
% Abstract
%*******************************************************
\pdfbookmark[1]{Abstract}{Abstract}
% \addcontentsline{toc}{chapter}{\tocEntry{Abstract}}

\begingroup

\let\clearpage\relax
\let\cleardoublepage\relax
\let\cleardoublepage\relax

\chapter*{Abstract}
Heart failure (HF) affects nearly a million people in the UK alone, and increases the risk
of cardiovascular diseases, stroke and death. At the whole-organ level, HF often manifests as impaired left ventricular (LV) contractile function. At the cellular level, LV contractile dysfunction is associated with altered sarcomere kinetics and disrupted calcium ($\Ca$) homeostasis. However, the link between cellular events and emerging pathological whole heart phenotypes is incompletely understood.

\vspace{0.2cm}
In this thesis, we aim at quantifying the translation of cellular pathophysiological mechanisms to the LV contractile function, and to elucidate the role of $\Ca$ and sarcomere dynamics in rat HF, with emphasis on the disease with preserved ejection fraction phenotype (HFpEF). We employed (Chapter~\ref{cha:chapter3}) a biophysically detailed 3D biventricular rat heart contraction mechanics model, which incorporated preload, afterload, fibre orientation, passive material properties, anatomy, $\Ca$ transient and sarcomere dynamics. The model cell-level function was controlled using a set of parameters (key regulators of the ionic, sarcomere contraction, tissue sub-models of the full multi-scale model). The model organ-level behaviour was described using a set of features characterising the LV volume and pressure transients and the corresponding pressure-volume (PV) loop.

\vspace{0.2cm}
We first (Chapter~\ref{cha:chapter4}) fitted the model to real biventricular geometries, volumetric and functional data from a sham-operated (SHAM) and an aortic-banded 6-weeks-post-surgery (AB) rats, respectively representative of the healthy control and diseased rat cohorts from an experimental study on AB rats (diastolic HF animal model). We then identified the key parameters that explained most of the variability observed in the considered LV features. Model fitting was performed using the history matching (HM) technique, while uncertainty quantification was performed using Sobol' global sensitivity analysis (GSA). These normally require a huge number of model evaluations to be performed. As the full forward model was too computationally expensive ($\sim 4-10$ hours), we made HM and GSA performance possible by replacing the input-to-output multi-scale map with fast-evaluating ($\sim 1$ second) probabilistic surrogates based on Gaussian process emulation (GPE). From now on, we will refer to the personalized (fitted) healthy SHAM rat model as ''the model". The model constituted the starting point of $3$ following studies.

\newpage
In the first study (Chapter~\ref{cha:chapter5}), we used the model to show that it is possible to map pharmacological compound modulations from the sarcomere through to whole heart function and back again. As a case study, we validated the omecamtiv mecarbil (OM) mechanisms of action across scales in the healthy rat heart. Preclinical force-calcium (FpCa) and LV haemodynamics data were used to constrain (GPE+HM) the parameter space to represent \textit{in silico} OM effects at the cellular level. The obtained space was then respectively mapped to LV function and FpCa curves to show that the model predictions are in qualitative agreement with the experimentally observed OM effects.

\vspace{0.2cm}
In the second study, we first (Chapter~\ref{cha:chapter6}) performed a validation against pharmacological channel blocking experimental data by showing that the model can predict the same literature compounds' effects on LV function. We then (Chapter~\ref{cha:chapter7}) used the model to generate a second model, representing the obese 20-week-old ZSF1 rat (HFpEF animal model). We then recovered (GPE+GSA+HM) the ZSF1 rat model back to the healthy state by perturbing different sub-groups of parameters to represent different strategies of recovery to identify potential pharmacological compounds' cellular targets for the treatment of HFpEF in rats.

\vspace{0.2cm}
In the third study (Chapter~\ref{cha:chapter8}), we used the model to demonstrate that changes in FpCa relationship do not uniquely map to observed changes in LV function and vice-versa. This result sheds a new light on pharmacological compounds' efficacy testing, which normally rely on monotonically/uniquely mapping the shifts induced in the FpCa curve to the corresponding evoked effect on the LV function.

\vspace{0.2cm}
We have built a virtual platform that can be used to efficiently test different pharmacological interventions and provide indication of/identify potential cellular pharmaceutical targets for "virtually'' treating HFpEF in rats. This was done by using computational models of cardiac mechanics and their probabilistic surrogates to quantify how normal/pathological cellular function is translated into normal/altered whole heart function. We have demonstrated the feasibility of applying Bayesian probabilistic techniques to small mammalian (rat) healthy and diseased cardiac models. This thesis constitutes an important step towards applications to more complex systems (human heart) for personalised medicine.

\endgroup

\vfill